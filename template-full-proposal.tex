\documentclass[11pt,twoside,a4paper]{article}
\usepackage{style}
\input{config}
\lhead{Ethics Proposal~--~\ApplicantName~--~\DateApplication~--~Version \HeadSha}

\begin{document}

\maketitle
\thispagestyle{fancy}

\section{Title of Project}

``\ProjectTitle''

\section{Name and Address of Applicant (Office address)}

\textbf{Name:} \ApplicantName{}

\textbf{Research Institution:} \ApplicantResearchInstitution{}

\textbf{Address:} \ApplicantAddress{}

\textbf{E-Mail Address:} \ApplicantEmailAddress{}

\textbf{Phone Number:} \ApplicantPhoneNumber{}

\textbf{Fax Number:} \ApplicantFaxNumber{}

\section{Context of Research}

This is an application for funding by >funding institution<.
An ethical review and statement by the ethics committee is required / not required.
It is a monocentric/multicentric study.
Other institutions involved in the study are:

\section{Subject and Procedure of Project}

\paragraph{Research Subject}

>Specify research objective, see Instructions for Submitting Applications (Paragraph 4)<.

\paragraph{Methods.}

>State main methods of investigation, e.g., measurement of reaction times, acquisition of EEG, completion of questionnaires<.

\paragraph{Sample size}

>Provide a detailed rationale for sample size including a power analysis if applicable, otherwise provide other detailed justification<

\paragraph{Experimental tasks.}

>Describe details of the experimental tasks here; what are the subjects instructed to do?<

\paragraph{Procedure.}

>Describe details of the study procedure here<

\paragraph{Physical stress.}

>e.g., Fatigue? Exertion? Invasive procedures? Medication? Drug testing?<

\paragraph{Mental stress.}

>e.g., aversive stimuli, negative experiences<

\paragraph{Disclosure of personal information.}

>What information is requested from the subjects?<

\paragraph{Deception and debriefing.}

>Will deception be used? When and how is it debriefed?
For intervention studies, is it made clear to subjects whether there is a control condition without intervention and how assignment to the intervention or control group is made?<

\section{Collection, Protected Storage, and Deletion of data}

\paragraph{Personal data.}

>e.g., collection of name, age, gender, regular medication, other personal data<


\paragraph{Data protection.}

>What data protection measures are provided? Pseudonymization (coding list) and subsequent anonymization; anonymization via personal code word; retention period for anonymized data<


\paragraph{Confidentiality / Obligation to data secrecy / Non-disclosure.}

>Are the researchers involved in the study legally bound to confidentiality, or have they been or will they be bound to data secrecy?
Do third parties (e.g., physicians or teachers) have to be released from their duty of confidentiality / their obligation to data secrecy by the participants in the context of the study?
Are participants in group settings explicitly asked to maintain confidentiality with regard to personal information disclosed by other participants?<

\paragraph{Deletion of the data.} 

>Information on data deletion with and without request. Period of retention of data that cannot be fully anonymized.<

\section{Recruitment and Compensation of Participants.}

\paragraph{Recruitment.}

>Insert information<

\paragraph{Sample from a database?}

>Details of the database, data protection officer must consent!<

\paragraph{Sample characteristics.}

>e.g., age, gender, population<

\paragraph{Inclusion and exclusion criteria.}

>List of inclusion and exclusion criteria. If exclusion criterion is pregnancy, outpatient pregnancy test required!<

\paragraph{Internet-based data collection.}

>How is compliance with inclusion and exclusion criteria ensured? Are contact persons available for the subjects in a timely manner?<

\paragraph{Participation copensation.}

>Compensation e.g., in money or participant-hours? Amount; method of payment<

\section{Voluntariness and Withdrawal of Consent}

\paragraph{Voluntariness.}

>Indicate measures to ensure voluntariness, e.g., participant information, time to decide whether to participate, avoidance of special benefits for participation<

\paragraph{Withdrawal.}

>Safeguarding the possibility of withdrawal at any time without disadvantages and the right to delete one's own data until the data is anonymized.<

\section{Handling of Abnormal Signs (e.g., EEG or MRT)}

\paragraph{Clarification.}

>How is the patient informed about abnormal findings, e.g.~in EEG, MRI or test diagnostic examinations?<

\paragraph{Restriction of participation.}

>Is it stated in the participant information that the subject can only participate in the examination if he/she agrees to be notified of any abnormal findings?
Is this consent obtained in the Informed Consent Form?
See templates for General Participant Information and Informed Consent.<

\section{Informed Consent}

\paragraph{Informedness.}

>Is the principle of complete information respected?
If not, what is the justification for incomplete information (deception) of the subjects?
How will the subjects be debriefed after the study (attach wording)?
Exactly what information will be given to subjects?
General and possibly specific participant information (e.g., for EEG, MRI, TMS studies) should be attached to the proposal; templates for these are available for download.<

\paragraph{Consent.}

>After informing the subjects, their consent is obtained.
Does the consent form contain all the necessary components (voluntary, informed, fully understood, possibility of withdrawal without disadvantages, signatures if consent form is printed)?
In addition, there may be other components, e.g., consent to specific research methods.
The consent form must be attached to the proposal; a template for this is available for download.<

\paragraph{Audio and video recordings.}

>A separate declaration of consent must be obtained if audio and video recordings are to be made; a template for this is available for download.<
If no: There are valid reasons for not involving the local ethics committee. I/we will explain the reasons to the Commission Chair in my/our cover letter.

\section{Disclosure of Possible Conflict of Interest of Applicant or Other Persons Involved in the Study}

\section{Previous Ethical Review Concerning this Project}

The research project described in this proposal has already been reviewed by an ethics committee. 

\YesNo{}

If so: The relevant ethics vote is attached to the application.

\section{Date and Signature}

\vspace{8ex}
\noindent\begin{tabular}{ll}
Hamburg, \today & Lennart Wittkuhn \\
\makebox[7cm]{\hrulefill} & \makebox[7cm]{\hrulefill}\\
Place and Date & Signature of Applicant\\[8ex]
\end{tabular}

\end{document}
